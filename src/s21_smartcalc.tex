\documentclass[12pt, letterpaper, twoside]{article}
\usepackage{cmap}
\usepackage[T1, T2A]{fontenc}
\usepackage[utf8]{inputenc}
\usepackage[english, russian]{babel}
\usepackage[a4paper,% размер страницы
text={180mm, 260mm},% ширина и высота текста
top=15mm, bottom=15mm, left=15mm, top=15mm]{geometry}
\usepackage{textcomp}

\title{SmartCalc v1.0}
\author{exegguto}
\date{Октябрь 2022}

\begin{document}
\maketitle

В данном проекте реализована на языке программирования Си с использованием структурного подхода расширенная версия обычного калькулятора, который можно найти в стандартных приложениях каждой операционной системы. Помимо базовых арифметических операций, как плюс-минус и умножить-поделить, калькулятор дополнен возможностью вычисления арифметических выражений с учетом приоритетов, а так же некоторыми математическими функциями (синус, косинус, логарифм и т.д.). Помимо вычисления выражений калькулятор так же поддерживает использование переменной x и построение графика соответствующей функции. Так же реализованы кредитный и депозитный калькуляторы.

Для корректной проверки данного проекта необходимо:

sudo apt install qt5-qmake

sudo apt install libqt5charts5 libqt5charts5-dev

sudo apt install xdg-utils

sudo apt install texlive-full

mac:

brew install qt6

brew link qt6 --force

brew install texlive

\begin{enumerate}
    \item Part 1. Реализация SmartCalc v1.0

    - **Арифметические операторы**:

    \begin{center}
      \begin{tabular}{ | c | c | }
        \hline
        Название оператора & Инфиксная нотация (Классическая) \\ 
        \hline\hline
        Скобки & (a + b) \\
        \hline
        Сложение & a + b \\
        \hline
        Вычитание & a - b \\
        \hline
        Умножение & a * b \\
        \hline
        Деление & a / b \\
        \hline
        Возведение в степень & a \^   b \\
        \hline
        Остаток от деления & a mod b \\
        \hline
        Унарный плюс & +a \\
        \hline
        Унарный минус & -a \\
        \hline
      \end{tabular}
    \end{center}

    >Обратите внимание, что оператор умножения содержит обязательный знак `*`. 

- **Функции**:

\begin{center}
  \begin{tabular}{ | c | c | }
    \hline
    Описание функции & Функция \\ 
    \hline\hline
    Вычисляет косинус & cos(x) \\
    \hline
    Вычисляет синус & asin(x) \\
    \hline
    Вычисляет тангенс & acos(x) \\
    \hline
    Вычисляет арккосинус & tan(x) \\
    \hline
    Вычисляет арксинус & asin(x) \\
    \hline
    Вычисляет арктангенс & atan(x) \\
    \hline
    Вычисляет квадратный корень & sqrt(x) \\
    \hline
    Вычисляет натуральный логарифм & ln(x) \\
    \hline
    Вычисляет десятичный логарифм & log(x) \\
    \hline
  \end{tabular}
\end{center}

- Область определения и область значения функций ограничиваются от -1000000 до 1000000
- Для построения графиков функции необходимо дополнительно указывать отображаемые область определения и область значения

    \item Part 2. Кредитный калькулятор
    \item Part 3. Депозитный калькулятор
  \end{enumerate}

\end{document}
